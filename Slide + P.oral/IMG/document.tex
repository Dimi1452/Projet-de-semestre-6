\documentclass[french]{article}
\usepackage[T1]{fontenc}
\usepackage[utf8]{inputenc}
\usepackage{lmodern}
\usepackage[a4paper]{geometry}
\usepackage{babel}
\usepackage{lipsum}
\usepackage{tikz}
\usetikzlibrary{shapes,arrows,positioning,circuits.ee.IEC}
\usepackage{circuitikz}
\usepackage{todonotes}
\usepackage {matlab-prettifier}
\usepackage{tabularx}
\usepackage{geometry}
\usepackage{soul}
\usepackage{xcolor}
\usepackage{rotating}
\begin{document}
\begin{figure}[h]
	\centering
	% Définition des styles des blocs
	\tikzstyle{function} = [rectangle, draw, minimum width=5cm, text centered, minimum height=1cm]
	\tikzstyle{condition} = [diamond, draw, aspect=2, minimum width=3cm, text centered]
	\tikzstyle{start}=[ellipse,draw,minimum width=3cm,minimum height=1.5cm, text centered]
	\tikzstyle{line} = [draw, -latex']
	\begin{tikzpicture}[node distance=1.5cm]
		
		% Création des nœuds
		\node[start](start1){Start};
	
		\node [function, below of=start1] (UserTE) {void UserTE(void)};
		\node [function,below of=UserTE] (function2) {Mesure position et écart};
		\node [condition, below=0.5cm of function2] (condition1) {Kp,Ki,Kd==0};
		\node [function, below=0.5cm of condition1] (function3) {void CalcParam(void)};
		\node [function, below right=0.75cm and 4cm of condition1] (function4) {void RegPID(void)};
		\node [function, below of=function4] (function5) {Limitation};
		\node [function, below of=function5] (function6) {Commande PWM};
		\node [start,below of=function6](start2){Stop};
		
		
		% Liaisons entre les nœuds
		\path [line] (start1) -- (UserTE);
		\path [line] (UserTE) -- (function2);
		\path [line] (function2) -- (condition1);
		\path [line] (condition1) -- node [near start, left] {oui} (function3);
		\path [line] (condition1) -- node [near start,below] {non} +(7,0) -| (function4.north);
		\path [line] (function3.south) -- ++(0,-0.5) -| ([xshift=-2cm] condition1.west) |- (condition1.north);
		\path [line] (function4) -- (function5);
		\path [line] (function5) -- (function6);
		\path [line] (function6) -- (start2);
		
	\end{tikzpicture}	

\end{figure}
\newpage
\begin{figure}[h]
	\centering
	% Définition des styles des blocs
	\tikzstyle{function} = [rectangle, draw, minimum width=5cm, text centered, minimum height=1cm]
	\tikzstyle{condition} = [diamond, draw, aspect=2, minimum width=3cm, text centered]
	\tikzstyle{start}=[ellipse,draw,minimum width=3cm,minimum height=1.5cm, text centered]
	\tikzstyle{line} = [draw, -latex']
	\begin{tikzpicture}[node distance=1.5cm]
		
		% Création des nœuds
		\node[start](start1){Interruption};
		\node [condition, below=1cm of start1] (condition1) {Bp1 pressed?};
		\node [condition, right=4cm of condition1] (condition2) {Bp2 pressed?};
		\node [function,below=0.5 of condition1] (function1) {Saut de consigne};
		\node [function,below=0.5 of condition2] (function2) {Kp,Ki,Kd=0};
		
	
		
		
		% Liaisons entre les nœuds
		\path [line] (start1) -- (condition1);
		\path [line] (start1.south) -| (condition2);
		\path [line] (condition1.east) -| node [near start, below] {non} (4,-1.75) -- (condition1.north);
		\path [line] (condition1) -- node [near start,left] {oui} (function1);
		\path [line] (condition2.east) -| node [near start, below] {non} (12,-1.75) -- (condition2.north);
		\path [line] (condition2) -- node [near start,left] {oui} (function2);

		
	\end{tikzpicture}	

\end{figure}
\end{document}
